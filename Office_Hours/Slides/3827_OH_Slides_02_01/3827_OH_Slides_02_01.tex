\documentclass{../slides}

\title{3827 OH}
\author{Eumin Hong (eh2890)}
\institute{Columbia University}
\date{February 1, 2022}

\begin{document}

\begin{frame}
    \titlepage
\end{frame}

\begin{frame}{Overview}
\begin{multicols}{2}
\tableofcontents
\end{multicols}
\end{frame}

\section{Introductions}
\subsection{About Me}
\begin{frame}{\secname: \subsecname}
    \begin{itemize}
        \item My name is Eumin Hong (UNI: eh2890)
        \item I am a junior in Columbia College studying computer science
        \item I am from New York
        \item This is my fourth time TAing for 3827 (second time with Professor Rubenstein)
        \item I took this course with Professor Kim in Fall 2020
        \begin{itemize}
            \item Took 1004, 3134, 3251 before 3827; took 3827 concurrently with 3203
        \end{itemize}
        \item Only need intro computer science course (how to code) and some high school math (mostly logic) to succeed
    \end{itemize}
\end{frame}

\subsection{About You}
\begin{frame}{\secname: \subsecname}
    \begin{itemize}
        \item Please introduce yourself and mention
        \begin{enumerate}
            \item Your name
            \item Year
            \item School
            \item Hometown
            \item Anything else that you would like to share
        \end{enumerate}
    \end{itemize}
\end{frame}

\section{Logistics}
\subsection{Course Grading}
\begin{frame}{\secname: \subsecname}
    \begin{itemize}
        \item Raw score $S$ formula:
        \begin{gather*}
            S = 0.1H + \max{\left(0.3M + 0.6F, 0.45M + 0.45F\right)} 
        \end{gather*}
        where $H$ is homework, $M$ is midterm, and $F$ is final
    \end{itemize}
\end{frame}

\subsection{Homework}
\begin{frame}{\secname: \subsecname}
    \begin{itemize}
        \item Homework will be submitted via Gradescope (entry code: 6P28R7)
        \item Scored out of $3$
        \item Relatively loose rubric -- check solutions or ask in OH for more detailed explanations
        \item Extensions are handled by me (not Professor Rubenstein)
        \begin{itemize}
            \item On the TA side, there is a spreadsheet with the latest possible submission date -- I cannot give out extensions past this date unless there is an emergency (in which case 
            \item Please contact me in advance about extensions -- psychologically, I prefer that you ask for an extension and not use it rather than have you ask me for an extension on the original due date
            \item I will grant extensions (that come in advance) in general, possible reasons include a lot of work/stress, interviews, upcoming tests/exams/essays, etc.
        \end{itemize}
    \end{itemize}
\end{frame}

\subsection{P-credit}
\begin{frame}{\secname: \subsecname}
    \begin{itemize}
        \item P-credit is a way to boost your grade, scored out of $3$
        \item Homeworks and exams count for less -- see \lstinline{3827_Lecture_00.pdf} for figures
        \item Obtain maximum P-credit by attending OH, participating, etc.
    \end{itemize}
\end{frame}

\section{OH Logistics}
\subsection{Structure}
\begin{frame}{\secname: \subsecname}
    \begin{itemize}
        \item Attendance will be taken via Zoom (either chat or participant history)
        \item After covering announcements (upcoming homeworks, exams, etc.), it is whatever you would like to do
        \begin{itemize}
            \item May have additional problems to highlight edge cases/tricks, or can end early
        \end{itemize}
        \item Make use of this time and prepare your questions in advance -- can be about previous/current homeworks, current course content, etc.
    \end{itemize}
\end{frame}

\subsection{Switching OH Sections}
\begin{frame}{\secname: \subsecname}
    \begin{itemize}
        \item For both permanent and temporary switches
        \item For different TA section: email TA whose section you would like to switch to, and \textbf{CC me}
        \begin{itemize}
            \item If you do not email any TA and attend other OH section, P-credit cannot be guaranteed
            \item If you miss my OH for the week and want to attend other OH section, email me first (to acknowledge missing OH) and then email TA and CC me
            \begin{itemize}
                \item No P-credit deduction for first couple times since stuff comes up
            \end{itemize}
            \item Email in advance
        \end{itemize}
        \item For my other section: email me
        \begin{itemize}
            \item To switch to my other section is not too big of an issue; I will probably be using the same Zoom link and the sections are back-to-back
        \end{itemize}
    \end{itemize}
\end{frame}

\subsection{OH Mode}
\begin{frame}{\secname: \subsecname}
    \begin{itemize}
        \item Either online or in-person
        \item Will definitely have OH in-person for MIPS (topic after the midterm)
        \item Pros and cons to either mode
        \item Will probably go with online OH for majority of semester for notes, convenience, and health
        \item OH will not be recorded, but blank slides will be on GitHub before each OH (ideally) and notes from OH will be on GitHub
    \end{itemize}
\end{frame}

\subsection{Communication}
\begin{frame}{\secname: \subsecname}
    \begin{itemize}
        \item If you want to contact me directly, send an email to \href{mailto:eh2890@columbia.edu}{eh2890@columbia.edu}
        \item Announcements will be via email
        \item Also open to a group messaging system for OH (will send announcements here as well if created)
        \begin{itemize}
            \item Options include Slack, GroupMe, etc.
            \item Will probably merge my OH sections
        \end{itemize}
    \end{itemize}
\end{frame}

\subsection{Feedback}
\begin{frame}{\secname: \subsecname}
    \begin{itemize}
        \item This semester, I will be trying anonymous feedback
        \item Form: \url{https://forms.gle/cnUmKVNYN7WvRbHA6}
    \end{itemize}
\end{frame}

\subsection{Resources}
\begin{frame}{\secname: \subsecname}
    \begin{itemize}
        \item OH GitHub: \url{https://github.com/eh2890/CSEE_W3827_S2022}
        \begin{itemize}
            \item Will have course lectures slides, OH slides, OH notes, etc. here
        \end{itemize}
        \item All OH Sections: \url{http://uribe.cs.columbia.edu/sched/assigned-slotorder.html}
        \item My OH Sections:
        \begin{itemize}
            \item Section \#11: Tuesday 3pm
            \item Section \#10: Tuesday 4pm
        \end{itemize}
        \item Course Google Calendar: \url{https://calendar.google.com/calendar/u/0/embed?src=16uc8kberc2b6dtltq6e49dv3k@group.calendar.google.com&ctz=America/New_York}
        \begin{itemize}
            \item Additional OH will be posted here and also announced
        \end{itemize}
    \end{itemize}
\end{frame}

\section{Homework 1}
\subsection{Problem 1}
\begin{frame}{\secname: \subsecname}
    Assume an architecture where all numbers are to be represented using 8 bits. What are the (base 10) values of the 8-bit binary numbers when interpreted using (i) unsigned, (ii) signed magnitude, (iii) 1’s complement, (iv) 2’s complement form:
    \begin{enumerate}[(a)]
        \item $00110011$
        \item $10000000$
        \item $11111111$
        \item $10011011$
        \item $10001010$
    \end{enumerate}
\end{frame}

\subsection{Problem 2}
\begin{frame}{\secname: \subsecname}
    Convert the following (base 10) numbers to their binary representation using 8-bit (i) signed magnitude, (ii) 1’s complement, (iii) 2’s complement forms:
    \begin{enumerate}[(a)]
        \item $-1$
        \item $-15$
        \item $-67$
        \item $-127$
    \end{enumerate}
\end{frame}

\subsection{Problem 3}
\begin{frame}{\secname: \subsecname}
    For each pair of $x$ and $y$ below, convert $x$ and $y$ to their 8-bit 2’s-complement forms, subtract $y$ from $x$ (so you need to negate $y$ and then add). Indicate whether an overflow occured (and show clearly how you know), and convert the solution back to base 10 (even when an overflow occured and the solution is wrong, i.e., your answer should fit in the 8-bit representation of that form.).
    \begin{enumerate}[(a)]
        \item $x = 10, y = -13$
        \item $x = 117, y = 35$
        \item $x = 117, y = -35$
        \item $x = -117, y = 35$
        \item $x = -117, y = 11$
    \end{enumerate}
\end{frame}

\subsection{Problem 4}
\begin{frame}{\secname: \subsecname}
    Given the bit pattern $1010\ 1101\ 0001\ 0000\ 0000\ 0000\ 0000\ 0010$, what does it represent, assuming that it is
    \begin{enumerate}[(a)]
        \item a two's complement integer?
        \item an unsigned integer?
        \item a single precision (32-bit) floating-point number?
    \end{enumerate}
\end{frame}

\subsection{Problem 5}
\begin{frame}{\secname: \subsecname}
    Represent the following numbers, given here in binary form, as floating point numbers using IEEE 754 floating- point standard representation:
    \begin{enumerate}[(a)]
        \item $11010.1110$
        \item $-11011.10$
        \item $0.000101$
        \item $-1.010101$
        \item $10000000010$
    \end{enumerate}
\end{frame}

\end{document}
